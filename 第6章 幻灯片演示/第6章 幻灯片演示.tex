\documentclass[11pt]{beamer}
\usepackage[heading=true]{ctex}
\usepackage[utf8]{inputenc}
\usepackage[T1]{fontenc}
\usepackage{lmodern}
\usepackage{listings}
\usetheme{CambridgeUS}

% listing code settings
\usepackage{listings}
\usepackage{xcolor}
\definecolor{backcolour}{rgb}{0.95, 0.95, 0.92}
\lstset{
	backgroundcolor=\color{backcolour},
	basicstyle=\ttfamily\footnotesize,
	tabsize=2, breaklines=true,
	frame=single
}

\begin{document}
	\title{\LaTeX 入门}
	\subtitle{第6章 幻灯片演示}
	\date{2022年8月26日}
	%\subject{}
	%	\setbeamercovered{transparent}
	%	\setbeamertemplate{navigation symbols}{}
	\begin{frame}[plain]
		\maketitle
	\end{frame}

\section{帧}

\begin{frame}[fragile]{帧}
\begin{lstlisting}
%\begin{frame}
	这是简单的一帧。
	帧里的内容是垂直居中的。
%\end{frame}
\end{lstlisting}
\begin{lstlisting}
%\begin{frame}
	\frametitle{标题}
	\framesubtitle{小标题}
	这是简单的一帧。
%\end{frame}
\end{lstlisting}
\end{frame}

\begin{frame}[fragile]{帧}
\begin{lstlisting}
%\begin{frame}{古中国数学}{定理发现}
中国在 3000 多年前就知道勾股数的概念,比古希腊更早一些。

《周髀算经》的记载:
\begin{itemize}
\item 公元前 11 世纪,商高答周公问:
\begin{quote}
勾广三,股修四,径隅五。
\end{quote}
\item 又载公元前 7--6 世纪陈子答荣方问,表述了勾股定理的一般形式:
\begin{quote}
若求邪至日者,以日下为勾,日高为股,勾股各自乘,并而开方除之,得邪至日。
\end{quote}
\end{itemize}
%\end{frame}
\end{lstlisting}
\end{frame}

\section{标题与文档信息}

\begin{frame}[fragile]{标题与文档信息}
\begin{lstlisting}
% beamer 导言区
\title{杂谈勾股定理}
\subtitle{数学史讲座之一}
\institute{九章学堂}
\author{张三}
\date{\today}
\subject{勾股定理}
\keywords{勾股定理, 历史}
\end{lstlisting}
\begin{lstlisting}
% 等价于 \maketitle
%\begin{frame}
	\titlepage
%\end{frame}
\end{lstlisting}
\end{frame}

\section{分节与目录}

\begin{frame}[fragile]{分节与目录}
\begin{lstlisting}
%\begin{frame}{目录}
	\tableofcontents
%\end{frame}
\section{勾股定理在古代}
...
\end{lstlisting}
\begin{lstlisting}
\part{引言}
%\begin{frame}
	\partpage
%\end{frame}
\end{lstlisting}
\begin{lstlisting}
% 导言区
\AtBeginSection[]{ % 空的可选项表示 \section* 前不加目录
%	\begin{frame}{本节提要}
		\tableofcontents[currentsection]
%	\end{frame}
}
\end{lstlisting}
\end{frame}
	
\end{document}