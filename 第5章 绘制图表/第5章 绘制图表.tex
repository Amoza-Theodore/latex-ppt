\documentclass[11pt]{beamer}
\usepackage[heading=true]{ctex}
\usepackage[utf8]{inputenc}
\usepackage[T1]{fontenc}
\usepackage{lmodern}
\usepackage{listings}
\usetheme{CambridgeUS}

% listing code settings
\usepackage{listings}
\usepackage{xcolor}
\definecolor{backcolour}{rgb}{0.95, 0.95, 0.92}
\lstset{
	backgroundcolor=\color{backcolour},
	basicstyle=\ttfamily\footnotesize,
	tabsize=2, breaklines=true,
	frame=single
}

\begin{document}
	\title{\LaTeX 入门}
	\subtitle{第5章 绘制图表}
	\date{2022年8月25日}
	%\subject{}
%	\setbeamercovered{transparent}
%	\setbeamertemplate{navigation symbols}{}
	\begin{frame}[plain]
		\maketitle
	\end{frame}
	
\part{\LaTeX 中的表格}	

\section{tabular和array}

\begin{frame}[fragile]{\LaTeX 中的表格}{tabular和array}
\begin{lstlisting}
\begin{tabular}{lcr}
left & center & right \\
本列左对齐 & 本列居中
	& 本列右对齐 \\
\end{tabular}
\end{lstlisting}
\begin{lstlisting}
\begin{tabular}{ll}
\bfseries 功能 & \bfseries 环境 \\
表格 & \ttfamily tabular \\
对齐 & \ttfamily tabbing \\
\end{tabular}
\end{lstlisting}
\end{frame}

\begin{frame}[fragile]{\LaTeX 中的表格}{tabular和array}
\begin{lstlisting}
\[
\begin{array}{r|r}
	\frac12 & 0 \\
	\hline
	0 & -\frac12 \\
\end{array}
\]
\end{lstlisting}
\begin{lstlisting}
\begin{tabular}[b]{c}
	上 \\ 中间 \\ 下
\end{tabular}
与底部对齐。
\end{lstlisting}
\end{frame}

\begin{frame}[fragile]{\LaTeX 中的表格}{tabular和array}
\begin{lstlisting}
\begin{tabular}{|rr|}
	\hline
	输入 & 输出 \\ \hline
	$-2$ & 4 \\
	0 & 0 \\
	2 & 4 \\
	\hline
\end{tabular}
\qquad
输入与输出有关系 $y = x^2$
\end{lstlisting}
\end{frame}

\section{表格单元的合并与分割}

\begin{frame}[fragile]{\LaTeX 中的表格}{表格单元的合并与分割}
\begin{lstlisting}
\begin{tabular}{|r|r|}
	\hline
	\multicolumn{2}{|c|}{成绩} \\ \hline
	语文 & 数学 \\ \hline
	87 & 100 \\ \hline
\end{tabular}
\end{lstlisting}
\begin{lstlisting}
% 导言区 \usepackage{multirow}
\begin{tabular}{|c|r|r|}
\hline
\multirow{2}*{姓名} &
\multicolumn{2}{c|}{成绩} \\ \cline{2-3}
	& 语文 & 数学 \\ \hline
张三 & 87  & 100 \\ \hline
\end{tabular}
\end{lstlisting}
\end{frame}

\begin{frame}[fragile]{\LaTeX 中的表格}{表格单元的合并与分割}
\begin{lstlisting}
% 导言区 \usepackage{makecell}
\begin{tabular}{|r|r|} \hline
\makecell{处理前\\数据} &
\makecell{处理后\\数据} \\ \hline
4934 & 8945 \\ \hline
\end{tabular}
\end{lstlisting}
\end{frame}

\begin{frame}[fragile]{\LaTeX 中的表格}{表格单元的合并与分割}
\begin{lstlisting}
% 导言区 \usepackage{multirow,makecell}
\begin{tabular}{|c|r|}
\hline
\multirowcell{3}{各科\\成绩} & 78 \\
\cline{2-2} & 82 \\ \cline{2-2}
& 86 \\ \hline
\end{tabular}
\end{lstlisting}
\end{frame}

\begin{frame}[fragile]{\LaTeX 中的表格}{表格单元的合并与分割}
\begin{lstlisting}
% 导言区 \usepackage{diagbox}
\begin{tabular}{|c|*{4}{c}|}
	\hline
	\diagbox{天干}{地支} & 子 & 丑 & 寅 & 卯 \\
	\hline
	甲 & 1 && 51 & \\
	乙 && 2 && 52 \\
	丙 & 13 && 3 & \\
	丁 && 14 && 4\\
	\hline
\end{tabular}
\end{lstlisting}
\end{frame}

\begin{frame}[fragile]{\LaTeX 中的表格}{表格单元的合并与分割}
\begin{lstlisting}
% 导言区 \usepackage{diagbox}
\begin{tabular}{|c|*{4}{c}|}
	\hline
	\diagbox{天干}{序号}{地支} & 子 & 丑 & 寅 & 卯 \\
	\hline
	甲 & 1 && 51 & \\
	乙 && 2 && 52 \\
	丙 & 13 && 3 & \\
	丁 && 14 && 4\\
	\hline
\end{tabular}
\end{lstlisting}
\end{frame}

\section{三线表与表线控制}

\begin{frame}[fragile]{\LaTeX 中的表格}{三线表与表线控制}
\begin{lstlisting}
\begin{tabular}{ccccc}
	\toprule
	序号 & 性别 & 年龄 & 身高/cm & 体重/kg \\
	\midrule
	1 & F & 14 & 156 & 42 \\
	2 & F & 16 & 158 & 45 \\
	3 & M & 14 & 162 & 48 \\
	4 & M & 15 & 163 & 50 \\
	\bottomrule
\end{tabular}
\end{lstlisting}
\end{frame}

\begin{frame}[fragile]{\LaTeX 中的表格}{三线表与表线控制}
\begin{lstlisting}
% 导言区 \usepackage{multirow,booktabs}
\begin{tabular}{*{6}{c}}
	\bottomrule
	\multirow{2}*{姓名} & \multicolumn{2}{c}{文科} &
	\multicolumn{2}{c}{理科} & \\
	\cmidrule(lr){2-3}\cmidrule(lr){4-5}\cmidrule(lr){6-6}
	\morecmidrules\cmidrule(lr){6-6}
	& 历史 & 文学 & 物理 & 化学 & 总评 \\
	\midrule
	张三 & A & A & B & A & A \\
	\bottomrule
\end{tabular}
\end{lstlisting}
\end{frame}

\part{插图与变换}

\section{graphicx与插图}

\begin{frame}[fragile]{插图与变换}{graphicx与插图}
\begin{lstlisting}
\includegraphics{ai.png}
\includegraphics[width=2em]{ai.png}
\includegraphics[height=1cm]{ai.png}
\includegraphics[scale=0.5]{ai.png}
\end{lstlisting}
\begin{lstlisting}
\graphicspath{{figures/}} % 本书的设置,图片在当前目录下的 figures 目录
\graphicspath{{pdf/}{png/}{jpg/}} % 按图片类型管理的
\end{lstlisting}
\end{frame}

\part{浮动体与标题控制}

\section{浮动体}

\begin{frame}[fragile]{浮动体与标题控制}{浮动体}
\begin{lstlisting}
\begin{figure}[htbp] % 允许各个位置
	\centering
	\includegraphics{ai.png}
\end{figure}
\begin{table} % 默认在页面顶部或单独一页
	\centering
	\begin{tabular}{|c|c|}
		\hline
		图形 & \verb=figure= 环境 \\
		\hline
		表格 & \verb=table= 环境 \\
		\hline
	\end{tabular}
\end{table}
\end{lstlisting}
\end{frame}

\begin{frame}[fragile]{浮动体与标题控制}{浮动体}
\begin{lstlisting}
\begin{figure}[htp]
	\centering
	\includegraphics{lion.eps}
	\caption[小狮子]{\TeX{} 系统的吉祥物——小狮子}\label{fig-lion}
	% 或作 \caption[小狮子]{\label{fig-lion}\TeX{} 系统的吉祥物——小狮子}
\end{figure}
\end{lstlisting}
\end{frame}

\section{并排与子图表}

\begin{frame}[fragile]{浮动体与标题控制}{并排与子图表}
\begin{lstlisting}
\begin{table}
	\centering
	\caption{并排的表格}
	\begin{tabular}{|c|c|}
		\hline 图 & 表 \\ \hline
	\end{tabular}%
	\qquad
	\begin{tabular}{|c|c|}
		\hline Figure & Table \\ \hline A & B \\ \hline
	\end{tabular}
\end{table}
\end{lstlisting}
\end{frame}

\begin{frame}[fragile]{浮动体与标题控制}{并排与子图表}
\begin{lstlisting}
\begin{figure}
\begin{minipage}[b]{.5\textwidth}
	\centering
	\includegraphics[width=.4\textwidth]{texlive-lion.pdf}
	\caption{\TeX\ Live 吉祥物狮子}
\end{minipage}%
\begin{minipage}[b]{.5\textwidth}
	\centering
	\begin{tabular}{|*{5}{c|}}
		\hline
		1996 & 1998 & 1999 & 2000 & 2001 \\ \hline
		2002 & 2003 & 2004 & 2005 & 2007 \\ \hline
		2008 & 2009 & 2010 & \dots & \\
		\hline
	\end{tabular}
	\caption{\TeX\ Live 的版本}
\end{minipage}
\end{figure}
\end{lstlisting}
\end{frame}

\section{浮动控制与float宏包}

\begin{frame}[fragile]{浮动体与标题控制}{浮动控制与float宏包}
\begin{lstlisting}
% \usepackage{float}
\begin{figure}[H]
	\centering
	\includegraphics[height=1cm]{lion.eps}
	\caption{不浮动的图表}
\end{figure}
\end{lstlisting}
\end{frame}

\end{document}