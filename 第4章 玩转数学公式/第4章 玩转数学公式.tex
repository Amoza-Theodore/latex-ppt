\documentclass[11pt]{beamer}
\usepackage{ctex}
\usepackage[utf8]{inputenc}
\usepackage[T1]{fontenc}
\usepackage{lmodern}
\usetheme{CambridgeUS}

% listing code settings
\usepackage{listings}
\usepackage{xcolor}
\definecolor{backcolour}{rgb}{0.95, 0.95, 0.92}
\lstset{
	backgroundcolor=\color{backcolour},
	basicstyle=\ttfamily\footnotesize,
	tabsize=2, breaklines=true,
	frame=single
}


\begin{document}
	\title{\LaTeX 入门}
	\subtitle{第4章 玩转数学公式}
	\date{2022年8月20日}
	\begin{frame}[plain]
		\maketitle
	\end{frame}
	
\part{数学模式概说}	

\begin{frame}[fragile]{数学模式概说}
\begin{lstlisting}
交换律是 $a+b=b+a$,如 $1+2=2+1=3$。

不能用 a+b=b+a,1+2=2+1=3。
\end{lstlisting}
\begin{lstlisting}
\usepackage{amsthm}
证毕符号:\qedsymbol 或 $\qedsymbol$。
\end{lstlisting}
\begin{lstlisting}
交换律是
\[ a+b=b+a, \]
如
\[
1+2=2+1=3.
\]
\end{lstlisting}
\end{frame}

\begin{frame}[fragile]{数学模式概说}
\begin{lstlisting}
\begin{equation}
a+b=b+a \label{eq:commutative}
\end{equation}
\end{lstlisting}
\begin{lstlisting}
\usepackage{amsmath}

$\text{被减数} - \text{减数} = \text{差}$

已知的变量有 $a$, $b$, $c$, $d$, $S$, $R$ 和 $T$。
\end{lstlisting}
\end{frame}

\part{数学结构}
\section{上标与下标}

\begin{frame}[fragile]{数学结构}{上标与下标}
\begin{lstlisting}
$A_{ij} = 2^{i+j}$
\end{lstlisting}

\begin{lstlisting}
$A_i^k = B^k_i$ \qquad
$K_{n_i} = K_{2^i} = 2^{n_i} = 2^{2^i}$ \qquad
$3^{3^{3^{\cdot^{\cdot^{\cdot^3}}}}}$
\end{lstlisting}
\begin{lstlisting}
$a = a’$, $b_0’ = b_0’’$,
${c’}^2 = (c’)^2$
\end{lstlisting}
\begin{lstlisting}
$A = 90^\circ$

\newcommand\degree{^\circ}
\end{lstlisting}
\end{frame}

\begin{frame}[fragile]{数学结构}{上标与下标}
\begin{lstlisting}
\[
	\max_n f(n) = \sum_{i=0}^n A_i
\]
\end{lstlisting}
\begin{lstlisting}
% 导言区 \DeclareMathOperator\dif{d\!}
\[ \int_0^1 f(t) \dif t
= \iint_D g(x,y) \dif x \dif y \]
\end{lstlisting}
\begin{lstlisting}
$\max_n f(n) = \sum_0^n A_i$
\end{lstlisting}
\begin{lstlisting}
% \usepackage{mathtools}
$\prescript{n}{m}{H}_i^j < L$
\end{lstlisting}
\end{frame}

\begin{frame}[fragile]{数学结构}{上标与下标}
\begin{lstlisting}
$\overset{*}{X}$ \qquad
$\underset{*}{X}$ \qquad
$\overset{*}{\underset{\dag}{X}}$
\end{lstlisting}
\begin{lstlisting}
$A_m{}^n$ 或 $A_m^{\phantom{m}n}$
\end{lstlisting}
\end{frame}

\begin{frame}[fragile]{数学结构}{上标与下标}
\begin{lstlisting}
% 导言区 \usepackage{mhchem}
醋中主要是 \ce{H2O},含有 \ce{CH3COO-}。

\ce{^{227}_{90}Th} 元素具有强放射性。
\end{lstlisting}
\begin{lstlisting}
\begin{equation}
\ce{2H2 + O2 ->[\text{燃烧}] 2H2O}
\end{equation}
\end{lstlisting}
\end{frame}

\section{上下画线与花括号}

\begin{frame}[fragile]{数学结构}{上下画线与花括号}
\begin{lstlisting}
$\overline{a+b} =
	\overline a + \overline b$ \\
$\underline a = (a_0, a_1, a_2, \dots)$
\end{lstlisting}
\begin{lstlisting}
$ \overline{\underline{\underline a}
+ \overline{b}^2} - c^{\underline n} $
\end{lstlisting}
\begin{lstlisting}
$\overleftarrow{a+b}$\\
$\overrightarrow{a+b}$\\
$\overleftrightarrow{a+b}$\\
$\underleftarrow{a-b}$\\
$\underrightarrow{a-b}$\\
$\underleftrightarrow{a-b}$
\end{lstlisting}
\end{frame}

\begin{frame}[fragile]{数学结构}{上下画线与花括号}
\begin{lstlisting}
$\overbrace{a+b+c} = \underbrace{1+2+3}$
\end{lstlisting}
\begin{lstlisting}
\[ ( \overbrace{a_0,a_1,\dots,a_n}
	^{\text{共 $n+1$ 项}} ) =
( \underbrace{0,0,\dots,0}_{n} , 1 ) \]
\end{lstlisting}
\begin{lstlisting}
\underbracket[<线宽>][<伸出高度>]{<内容>}
\overbracket[<线宽>][<伸出高度>]{<内容>}

\[ \underbracket{\overbracket{1+2}+3}_3 \]
\end{lstlisting}
\end{frame}

\section{分式}

\begin{frame}[fragile]{数学结构}{分式}
\begin{lstlisting}
\[
\frac 12 + \frac 1a = \frac{2+a}{2a}
\]
\end{lstlisting}
\begin{lstlisting}
通分计算 $\frac 12 + \frac 1a$
得 $\frac{2+a}{2a}$
\end{lstlisting}
\begin{lstlisting}
\[ \frac{1}{\frac 12 (a+b)}
= \frac{2}{a+b} \]
\end{lstlisting}
\begin{lstlisting}
% \usepackage{xfrac}
区别 $\sfrac 1a + b$ 和 $1/(a+b)$
\end{lstlisting}
\begin{lstlisting}
\[
(a+b)^2 = \binom 20 a^2
+ \binom 21 ab + \binom 22 b^2
\]
\end{lstlisting}
\end{frame}

\section{根式}

\begin{frame}[fragile]{数学结构}{根式}
\begin{lstlisting}
$\sqrt 4 = \sqrt[3]{8} = 2$
\end{lstlisting}
\begin{lstlisting}
\[
\sqrt[n]{\frac{x^2 + \sqrt 2}{x+y}}
\]
\end{lstlisting}
\begin{lstlisting}
\[
(x^p+y^q)^{\frac{1}{1/p+1/q}}
\]
\end{lstlisting}
\begin{lstlisting}
$\sqrt b \sqrt y$ \qquad
$\sqrt{\mathstrut b} \sqrt{\mathstrut y}$
\end{lstlisting}
\end{frame}

\section{矩阵}

\begin{frame}[fragile]{数学结构}{矩阵}
\begin{lstlisting}
\[
A = \begin{pmatrix}
	a_{11} & a_{12} & a_{13} \\
	0 & a_{22} & a_{23} \\
	0 & 0 & a_{33}
\end{pmatrix}
\]
\end{lstlisting}
\begin{lstlisting}
\[
A = \begin{bmatrix}
	a_{11} & \dots & a_{1n} \\
	& \ddots & \vdots \\
	0 & & a_{nn}
\end{bmatrix}_{n\times n}
\]
\end{lstlisting}
\end{frame}

\begin{frame}[fragile]{数学结构}{矩阵}
\begin{lstlisting}
\[ \begin{pmatrix}
	1 & \frac 12 & \dots & \frac 1n \\
	\hdotsfor{4} \\
	m & \frac m2 & \dots & \frac mn
\end{pmatrix} \]
\end{lstlisting}
\begin{lstlisting}
\[ \begin{pmatrix}
\begin{matrix} 1&0\\0&1 \end{matrix}
& \text{\Large 0} \\
\text{\Large 0} &
\begin{matrix} 1&0\\0&-1 \end{matrix}
\end{pmatrix} \]
\end{lstlisting}
\end{frame}

\begin{frame}[fragile]{数学结构}{矩阵}
\begin{lstlisting}
复数 $z = (x,y)$ 也可用矩阵 \begin{math}
\left( \begin{smallmatrix}
x & -y \\ y & x
\end{smallmatrix} \right)
\end{math} 来表示。
\end{lstlisting}
\begin{lstlisting}
\[
\sum_{\substack{0<i<n \\ 0<j<i}} A_{ij}
\]
\end{lstlisting}
\end{frame}

\part{符号与类型}

\begin{frame}[fragile]{符号与类型}
\begin{center}	$\dots$ \end{center}
\end{frame}

\part{多行公式}

\section{罗列多个公式}

\begin{frame}[fragile]{多行公式}{罗列多个公式}
\begin{lstlisting}
\begin{gather}
a+b = b+a \\
ab = ba
\end{gather}
\end{lstlisting}
\begin{lstlisting}
\begin{gather*}
3+5 = 5+3 = 8 \\
3\times 5 = 5\times 3
\end{gather*}
\end{lstlisting}
\begin{lstlisting}
\begin{gather}
3^2 + 4^2 = 5^2 \notag \\
5^2 + 12^2 = 13^2 \notag \\
a^2 + b^2 = c^2
\end{gather}
\end{lstlisting}
\end{frame}

\begin{frame}[fragile]{多行公式}{罗列多个公式}
\begin{lstlisting}
\begin{align}
	x &= t + \cos t + 1 \\
	y &= 2\sin t
\end{align}
\end{lstlisting}
\begin{lstlisting}
\begin{align*}
x &= t & x &= \cos t
& x &= t \\
y &= 2t & y &= \sin(t+1) & y &= \sin t
\end{align*}
\end{lstlisting}
\end{frame}

\begin{frame}[fragile]{多行公式}{罗列多个公式}
\begin{lstlisting}
\begin{align*}
	& (a+b)(a^2-ab+b^2) \notag \\
={}	& a^3 - a^2b + ab^2 + a^2b
	  - ab^2 + b^2 \notag \\
={}	& a^3 + b^3 \label{eq:cubesum}
\end{align*}
\end{lstlisting}
\begin{lstlisting}
\begin{align*}
x^2 + 2x &= -1
\intertext{移项得}
x^2 + 2x + 1 &= 0
\end{align*}
\end{lstlisting}
\end{frame}

\begin{frame}[fragile]{多行公式}{罗列多个公式}
\begin{lstlisting}
设 $G$ 是一个带有运算 $*$ 的集合,则 $G$ 是\emph{群},当且仅当:
\begin{subequations}\label{eq:group}
\begin{alignat}{2}
\forall a,b,c &\in G, &\qquad (a*b)*c &= a*(b*c);\label{subeq:assoc}\\
\exists e, \forall a &\in G, & e*a &= a; \\
\forall a, \exists b &\in G, & b*a &= e.
\end{alignat}
\end{subequations}
式~\eqref{eq:group} 的三个条件中,\eqref{subeq:assoc}~又称为结合律。
\end{lstlisting}
\end{frame}

\section{拆分单个公式}

\begin{frame}[fragile]{多行公式}{拆分单个公式}
\begin{lstlisting}
\begin{equation} \begin{split}  % 不产生编号
\cos 2x &= \cos^2 x - \sin^2 x \\
		&= 2\cos^2 x - 1
\end{split} \end{equation}
\end{lstlisting}
\begin{lstlisting}
\begin{equation}\label{eq:trigonometric}
\begin{split}
	\frac12 (\sin(x+y) + \sin(x-y))
	&= \frac12(\sin x\cos y + \cos x\sin y) \\
	&\quad + \frac12(\sin x\cos y - \cos x\sin y) \\
	&= \sin x\cos y
\end{split}
\end{equation}
\end{lstlisting}
\end{frame}

\section{将公式组合成块}

\begin{frame}[fragile]{多行公式}{将公式组合成块}
\begin{lstlisting}
\[ \left. \begin{gathered}
	S \subseteq T \\
	S \supseteq T
\end{gathered} \right\}
\implies S = T \]
\end{lstlisting}
\end{frame}
\begin{frame}[fragile]{多行公式}{将公式组合成块}
\begin{lstlisting}
\begin{equation}\label{eq:trinary}
\begin{aligned} x+y &= -1 \\ x+y+z &= 2 \\ xyz &= -6 \end{aligned}
\implies
\begin{aligned} x+y &= -1 \\ xy &= -2 \\ z &= 3 \end{aligned}
\implies
\begin{alignedat}{3}
		    x &= 1,  &\quad y &= -2, &\quad z &= 3 \\
\text{或\ } x &= -2, &      y &= 1, &       z &= 3
\end{alignedat}
\end{equation}
\end{lstlisting}
\end{frame}

\end{document}