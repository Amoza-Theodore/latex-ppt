\documentclass[11pt]{beamer}
\usepackage{ctex}
\usepackage[utf8]{inputenc}
\usepackage[T1]{fontenc}
\usepackage{lmodern}
\usetheme{CambridgeUS}

\usepackage{listings}

\usepackage{xcolor}

%New colors defined below
\definecolor{codegreen}{rgb}{0,0.6,0}
\definecolor{codegray}{rgb}{0.5,0.5,0.5}
\definecolor{codepurple}{rgb}{0.58,0,0.82}
\definecolor{backcolour}{rgb}{0.95,0.95,0.92}

%Code listing style named "mystyle"
\lstdefinestyle{mystyle}{
	backgroundcolor=\color{backcolour}, commentstyle=\color{codegreen},
	keywordstyle=\color{magenta},
	numberstyle=\tiny\color{codegray},
	stringstyle=\color{codepurple},
	basicstyle=\ttfamily\footnotesize,
	breakatwhitespace=false,         
	breaklines=true,                 
	captionpos=b,                    
	keepspaces=true,                 
	numbers=left,                    
	numbersep=5pt,                  
	showspaces=false,                
	showstringspaces=false,
	showtabs=false,                  
	tabsize=2
}

%"mystyle" code listing set
\lstset{style=mystyle}


\begin{document}
	%\author{}
	\title{\LaTeX 入门}
	\subtitle{第3章 自动化工具}
	%\logo{}
	%\institute{}
	\date{2022年8月16日}
	%\subject{}
	%\setbeamercovered{transparent}
	%\setbeamertemplate{navigation symbols}{}
	\begin{frame}[plain]
		\maketitle
	\end{frame}
	
\part{目录}

\section{目录和图表目录}

\begin{frame}[fragile]{目录}{目录和图表目录}
\begin{lstlisting}
\documentclass{article}
\begin{document}
	\tableofcontents
	\section{Foo}
	\subsection{blah}
	\section{Bar}
\end{document}
\end{lstlisting}
\end{frame}

\section{控制目录内容}

\begin{frame}[fragile]{目录}{控制目录内容}
\begin{lstlisting}
\addcontentsline{toc}{section}{Title}
\maketitle
\tableofcontents
\end{lstlisting}

\begin{lstlisting}
% 把目录、参考文献等条目加入目录
\usepackage{tocbibind}
\end{lstlisting}
\end{frame}

\part{交叉引用}

\section{电子文档与超链接}

\begin{frame}[fragile]{交叉引用}{电子文档与超链接}
\begin{lstlisting}
\usepackage{hyperref}

\url{http://bbs.ctex.org/forum.php?mod=viewthread&tid=48244#pid337079}
\href{http://bbs.ctex.org/}{CTeX 论坛}
\end{lstlisting}
\end{frame}

\part{BIBTEX与文献数据库}

\section{BIBTEX基础}

\begin{frame}[fragile]{BIBTEX与文献数据库}{BIBTEX基础}
\begin{lstlisting}
% tex.bib 中的一条
@BOOK{mittelbach2004,
	title = {The {{\LaTeX}} Companion},
	publisher = {Addison-Wesley},
	year = {2004},
	author = {Frank Mittelbach and Michel Goossens},
	series = {Tools and Techniques for Computer Typesetting},
	address = {Boston},
	edition = {Second}
}
\end{lstlisting}
\end{frame}

\begin{frame}[fragile]{BIBTEX与文献数据库}{BIBTEX基础}
\begin{lstlisting}
\bibliographystyle{alpha}  % style on book

\TeX{} and \LaTeX{} see \cite{knuthtex1986}, \cite{lamport1994}.
\nocite{mittelbach2004}

\bibliography{tex}% 表示使用的数据库是 tex.bib
\end{lstlisting}

\begin{lstlisting}
xelatex foo.tex
bibtex foo.aux
xelatex foo.tex
xelatex foo.tex
\end{lstlisting}
\end{frame}

\section{用natbib定制文献格式}

\begin{frame}[fragile]{BIBTEX与文献数据库}{用natbib定制文献格式}
\begin{lstlisting}
% 导言区
% \usepackage{natbib}
% \bibliographystyle{plainnat}
% 正文
在 \citet{lamport1994} 中提到了利用 \BibTeX{} 自动处理文献的方式,在另一本书 \citep{mittelbach2004} 中则有进一步的格式与工具的说明
\end{lstlisting}

\begin{lstlisting}
\usepackage[numbers,square]{natbib}
\end{lstlisting}
\begin{lstlisting}
\usepackage[numbers,sort&compress]{natbib}
\end{lstlisting}
\end{frame}

\end{document}